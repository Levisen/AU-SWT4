% !TeX root = ../../Handin3.tex
\subsection{Erfaringer med arbejdsfordelingen i praksis}

I forbindelse med udarbejdelse af opgaven og fokus på at arbejde continuerbart, sammen på projektet
har opgaverne været lidt fordelt ud. I starten fik vi lavet nogle Unit Tests, 
helt konkret til udregning af hastigheden og kursen for de de enkelte fly.
I den forbindelse fik vi oprettet en masse test cases på baggrund af testdata, for eventuelle fly, 
som var nogle af de Unit test, vi havde fået oprettet inden implementeringen af selve systemmet.

Beregningerne til Assert funktionerne, som udregnings funktionerne i FlightCourseCalculator og FlightVelocityCalculator, 
skulle teste imod. Er beskrevet i Figur \ref{data_unittest} og \ref{resultat_unittest}.

\insertfigure{test_data.PNG}{0.9}{data_unittest} {Beregninger for udfærdigelse af Unit Test, til FlightCourseCalculator og FlightVelocityCalculator funktionerne}

\insertfigure{test_result.PNG}{0.9}{resultat_unittest} {Resultater til at teste med Assertions i Unit Test, til FlightCourseCalculator og FlightVelocityCalculator funktionerne}

Vi fik derfor implementeret de resterende Unittest. Hvor de fleste af testene og kørte fint, 
men for ca. 70\% af testene var der fejl i testene, 
grundet vi ikke havde taget højde for decimalberegninger som var bedre en dem excel kunne udregne for os. Vi lavet en løsning på test funktionen, 
der endte med at afrunde, da vi stadig betragne vores udregninger for korrekte. Heldigvis blev der grønt efterfølgende og testene gik igennem.