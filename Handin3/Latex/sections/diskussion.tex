% !TeX root = ../Handin3.tex

\section{Reflektioner og Diskussion}
Vi er som udgangpunkt tilfredse med vores tilgang til at lave et testbart design, men det har nok taget for meget af vores fokus, 
da vi undervurderede opgavens omfang i første omgang. 
Det har været lærerigt at udvikle et system med en test-fokuseret løsning, 
hvilket virkelig har vist sine fordele, for hvilke ting man kan undgå af fejl.

Vi har stadig gjort brug af mere familære testmetoder som konsoludskrifter m.m., for at teste diverse aspekter af applikationen undervejs. 
Dette er oftest gået hurtigere end at implementere unit tests, 
men kan på ingen måde hamle op med unit tests ift skalérbarhed.

Dertil kommer kommunikationen mellem de funktioner hvert interface definerer, 
som i vores design foregår via events.

Fremadrettet bør vi udnytte interfaces mere mellem klasserne, 
der skal snakke sammen. På den måde kan vi sikre os Unit tests til vores interfaces, 
og realisere via fakes, hvordan vi gerne vil have det til at blive, uden reelt at have implementeret noget i forvejen.

Derudover har vi efter den sidste del af ATM, fået erfaring med at implementere integrationstestene, 
samt at udnytte et vigtigt værktøj, som dotCover, med flere, der virkelig kan give projektet et belæg for at lave Unit Tests og udnytte test metoder, fra starten..

Erfaringen er, at på et tispunkt, vil et projekt blive fort stort. 
Dermed er det vigtigt at udnytte værktøjerne og gøre betænkningstiden og designovervejelsesperoiden mere overskuelige, 
ved allerede at tænke i tests og dele projektet op fra begyndelsen.
