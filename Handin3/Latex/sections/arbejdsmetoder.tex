
\section{Arbejdsmetoder}

\subsection{Continuous Integration og Versionstyring}

Vi har i forbindelse med projektet udnyttet Continiuous Integration, 
der bygger på at udnytte mulighedden for at teste systemet undervejs, 
mens løsningen til projektet implementeres. 
I den forbindelse har vi udnyttet Jenkins serveren, samt inkluderet 
et NUnit.Testaadaptor framework, til at teste nogle metoder direkte i VS2017. 
I forbindelse med implementeringen, havde vi udfordringer med projektet 
og gentagende ændringer for strukturen, inden implementeringen blev færdig. 
Desværre medførte det også, at vi ikke fortsatte med udnytte teorien bag Continues 
Integration. I forbindelse med test af projektet oplevede vi at data ikke blev 
behandlet korrekt, men godt kunne kompilere, hvilket er en af årsagerne til 
at vi fremadrettet bør implementere unit tests undervejs.
Vi har benyttet git til versionstyring, 

Det testcentriske design har også vist sig at have en fordel når det kom til at arbejde 
flere på samme projekt, i og med at grænsefladerne er mere veldefinerede. 
Vi har derfor nemmere ved at definere de Unit test som er nødvendige, 
for at vi for defineret de tests som reeltset er nødvendige for systemet. 

Alligevel bør man ideelt set teste alt arbejdet løbende, også selvom det er noget generelt.
Hvilken vi i løbet af udarbejdelse af projektet også fandt ud af.
Som ugangspunkt, har vi derfor udnyttet vores interfaces til at teste vores klasser ud fra. 
På den måde har vi kunnet holde en struktur igennem arbejdsprocessen, 
der har medført at projektet stadig har været overskueligt at arbejde med.

På samme måde har vi udnyttet metoden med at Unit teste gennem vores interfaces, når vi skulle teste events, mellem de forskellige dele af systemet.