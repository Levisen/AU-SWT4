
\section{Arbejdsmetoder}

\subsection{Continuous Integration og Versionstyring}

Vi har i forbindelse med projektet udnyttet Continiuous Integration, 
der bygger på at udnytte mulighedden for at teste systemet undervejs, 
mens løsningen til projektet implementeres. 
I den forbindelse har vi udnyttet Jenkins serveren fra sidst samt inkluderet 
et NUnit.Testaadaptor framework, til at teste nogle metoder direkte i VS2017. 
I forbindelse med implementeringen, havde vi udfordringer med projektet 
og gentagende ændringer for strukturen, inden implementeringen blev færdig. 
Desværre medførte det også at vi ikke fortsatte med udnytte teorien bag Continues 
Integration. I forbindelse med test af projektet oplevede vi at data ikke blev 
behandlet korrekt, men godt kunne kompilere, hvilket er en af årsagerne til 
at vi fremadrettet bør implementere unit tests undervejs.
Vi har benyttet git til versionstyring, 

Det testcentriske design har også vist sig at have en fordel når det kom til at arbejde 
flere på samme projekt, i og med at grænsefladerne er mere veldefinerede. 
Vi har derfor nemmere ved at definere de Unit test som er nødvendige, 
for at vi for implementeret de tests som er nødvendige.
Som ugangspunkt, har vi derfor udnyttet Vores interfaces og events til 
at oprette vores unit tests ud fra. På den måde vil det fremadrettet være mere 
gennemskueligt når vi skal test projekter.